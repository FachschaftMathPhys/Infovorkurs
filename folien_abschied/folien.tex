\documentclass{beamer}
\usepackage[utf8]{inputenc}
\usepackage[T1]{fontenc}
\usetheme{Warsaw}
%\usecolortheme{albatross}
\usepackage{amsmath,nicefrac,amsthm,txfonts,amssymb}
\usepackage{graphicx}
\usepackage[overlay,absolute]{textpos}
\setlength{\TPHorizModule}{1mm}
\setlength{\TPVertModule}{1mm}

\newcommand{\Ra}{\Rightarrow}
\newcommand{\R}{\mathbb{R}}
\newcommand{\Enc}{\mathrm{Enc}}

\begin{document}

\title{Programmiervorkurs - Abschlussworte}
\author{Jakob Schnell \& Friedrich Schwedler \& Patrick Dammann}
\date{\today}

\begin{frame}
\titlepage
\end{frame}

\begin{frame}
    \begin{center}
        \Huge Ein paar Worte zum Abschied
    \end{center}
    \pause\begin{center}
        \Huge Zuerst: Danke fürs Teilnehmen, bitte gebt feedback!
    \end{center}
\end{frame}

\begin{frame}
    \frametitle{Das würden wir uns wünschen, habt ihr „gelernt“}
    \begin{itemize}
        \pause\item Ein Computer ist keine schwarze Magie
        \pause\item Eine Konsole ist keine schwarze Magie
        \pause\item Programmieren ist keine schwarze Magie
        \pause\item Ihr wisst, wo ihr anfangt, wenn die Aufgabe ist „schreibt
            ein Programm, das\dots“
        \pause\item Ihr entwickelt Spaß daran, Programmieraufgaben zu lösen
        \pause\item Ihr wisst, was ihr tun könnt, wenn etwas nicht funktioniert
    \end{itemize}
\end{frame}

\begin{frame}
    \frametitle{Where do we go, from here?}

    \begin{itemize}
        \pause\item Der Vorkurs ist nicht dazu gedacht, unbedingt abgeschlossen
            zu werden
        \pause\item Nächste Woche beginnt der „richtige“ Vorkurs (Montag, 10
            Uhr, INF 306, HS 1)
        \pause\item Wenn ihr unbedingt zu Hause weiter arbeiten wollt, gibt es
            mehrere Möglichkeiten\dots
        \pause\item (\dots die Folien stelle ich online. Ihr müsst nicht
            mitschreiben)
        \pause\item Alle Materialien und Folien findet ihr (heute abend
            spätestens auch aktuell) auf \url{http://mathphys.info/vorkurs/programmier}
    \end{itemize}
\end{frame}

\begin{frame}
    \frametitle{Codeblocks}

    \begin{itemize}
        \pause\item „Integrated Development Environment“ (IDE)
        \pause\item Integriert den Compiler, einen debugger, einen Editor und
            mehr
        \pause\item Ist die „offiziell empfohlene“ IDE in der Einführung in die
            praktische Informatik
        \pause\item Läuft unter Windows, Linux und Mac
    \end{itemize}
\end{frame}

\begin{frame}
    \frametitle{Linux Install Party}

    \begin{itemize}
        \pause\item Workshop im Vorkurs
        \pause\item Bringt einen Laptop -- wir installieren mit euch Linux
        \pause\item Hier ist ein „Debian Wheezy mit xfce“ installiert
        \pause\item Ihr braucht \texttt{g++} und \texttt{gdb} zusätzlich
        \pause\item Linux ist toll!
        \pause\item Aber gewöhnungsbedürftig.
        \pause\item Aber es lohnt sich ;)
    \end{itemize}
\end{frame}

\begin{frame}
    \frametitle{Wubi}

    \begin{itemize}
        \pause\item Installiert euch ein Ubuntu-Linux, das ihr aus Windows
            starten könnt
        \pause\item Nutzt Google für Hilfe und Installationsanleitung
        \pause\item Ubuntu nutzt einen Unity-Desktop -- der sieht ganz anders
            aus als hier, ihr müsst mal schauen, wie ihr klar kommt
        \pause\item Ihr braucht mindestens folgende Zusatzpakete (nutzt
            Google):
            \texttt{build-essential}, \texttt{g++}, \texttt{gdb}
    \end{itemize}
\end{frame}

\begin{frame}
    \begin{center}
        \Huge Fragen?
    \end{center}
    \pause\begin{center}
        \Huge Tschüss!
    \end{center}
\end{frame}


\end{document}
