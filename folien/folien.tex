\documentclass{beamer}
\usepackage[utf8]{inputenc}
\usepackage[T1]{fontenc}
\usetheme{Warsaw}
%\usecolortheme{albatross}
\usepackage{amsmath,nicefrac,amsthm,txfonts,amssymb}
\usepackage{graphicx}
\usepackage[overlay,absolute]{textpos}
\setlength{\TPHorizModule}{1mm}
\setlength{\TPVertModule}{1mm}

\newcommand{\Ra}{\Rightarrow}
\newcommand{\R}{\mathbb{R}}
\newcommand{\Enc}{\mathrm{Enc}}

\begin{document}

\title{Willkommen beim Vorkurs}
\author{Patrick Dammann \& Johannes Visintini}
\date{\today}

\begin{frame}
\titlepage
\end{frame}

\begin{frame}
    \begin{center}
        \Huge Eine kurze Einführung
    \end{center}
    \begin{center}
        \pause\Huge Weil das doch ein bisschen anders laufen wird, als ihr es gewohnt seid\dots
    \end{center}
\end{frame}

\begin{frame}
    \frametitle{Worum es geht}
    \begin{itemize}
        \item Ihr habt (fast) noch nie programmiert
        \pause\item Ihr müsst alle programmieren
        \pause\item Eine Woche Vorbereitung
        \pause\item Programmieren kann man nicht „beibringen“, man muss es „lernen“
        \pause\item $\Ra$ Wir bringen euch bei, Programmieren zu lernen
    \end{itemize}
\end{frame}

\begin{frame}
    \frametitle{Was wir von euch erwarten}
    \begin{itemize}
        \item Ihr seid nicht hier, um ein Zertifikat zu bekommen
        \pause\item Ihr wollt euch den Programmiervorkurs nicht in den Lebenslauf schreiben
        \pause\item Ihr wollt keine Zeit totschlagen
        \pause\item Ihr erwartet keinen „Trichter“ von uns
        \pause\item $\Ra$ Ihr seid hier, um zu lernen (aktiv)
    \end{itemize}
\end{frame}

\begin{frame}
    \frametitle{Was ihr von uns erwarten könnt}
    \begin{itemize}
        \item Wir geben euch Hilfsmittel um zu lernen
        \pause\item Wir helfen euch, diese Hilfsmittel zu interpretieren
        \pause\item Wir helfen euch, weitere Informationen zu finden
        \pause\item Wir erklären euch Dinge, die nicht im Skript stehen
        \pause\item Wir helfen euch, wenn etwas nicht funktioniert
        \pause\item Wir loben euch, wenn ihr mit eurem Können angeben wollt ;)
    \end{itemize}
\end{frame}

\begin{frame}
    \frametitle{Wie das Skript funktioniert}
    \begin{itemize}
        \item Auf eurem Desktop liegt eine vorkurs.pdf
        \pause\item Das Skript ist aufgeteilt in \emph{Lektionen}
        \pause\item Ihr arbeitet die in eurem eigenen Tempo durch
        \pause\item Jede Lektion hat einen \emph{Theorie-}, einen
            \emph{Praxis-} und einen \emph{Spiel}teil
        \pause\item \emph{Theorie}: Wall of Text. Aufmerksam lesen, ggf. später
            als Referenz nutzen
        \pause\item \emph{Praxis}: (Relativ) einfache Aufgaben, lassen sich
            ohne großes Verständnis machen
        \pause\item \emph{Spiel}: Herausforderndere, freiere Aufgaben. Hier
            sollt ihr selbst erkunden
    \end{itemize}
\end{frame}

\begin{frame}
    \frametitle{Das sollt ihr „lernen“}
    \begin{itemize}
        \item Ein Computer ist keine schwarze Magie
        \pause\item Eine Konsole ist keine schwarze Magie
        \pause\item Programmieren ist keine schwarze Magie
        \pause\item Ihr wisst, wo ihr anfangt, wenn die Aufgabe ist „schreibt
            ein Programm, das\dots“
        \pause\item Ihr entwickelt Spaß daran, Programmieraufgaben zu lösen
        \pause\item Ihr wisst, was ihr tun könnt, wenn etwas nicht funktioniert
    \end{itemize}
\end{frame}

\begin{frame}
    \frametitle{Organisatorisches}
    \begin{itemize}
        \item 50 Accounts: \texttt{vkmp\#\#\#}
        \pause\item \#\#\# ist die Nummer eures Computers
        \pause\item Initiales Passwort: \texttt{M4tphy\$2014} (ändern)
        \pause\item Merkt euch eure Nummer! Sie bleibt euer Account (auch, wenn
            ihr morgen woanders sitzt)
        \pause\item Tragt euch in die Liste ein
    \end{itemize}
\end{frame}

\begin{frame}
    \begin{center}
        \Huge Einführung
    \end{center}
    \begin{center}
        \pause\Huge Fragen?
    \end{center}
\end{frame}

\end{document}
