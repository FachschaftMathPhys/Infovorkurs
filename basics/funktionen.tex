\lesson{Funktionen}

Wir haben den Begriff schon ein paarmal genannt und auch, wenn ihr es
vielleicht nicht gemerkt habt, sie waren schon die ganze Zeit vor eurer Nase:
Funktionen.

Eine Funktion ist im Wesentlichen eine Möglichkeit, eine Folge von Anweisungen
zu bündeln und isolieren. Wir haben schon einmal ein Programm gesehen, mit dem
wir testen können, ob eine Zahl eine Primzahl ist, oder nicht. Viele Vorteile
von Funktionen, lassen sich an diesem Beispiel illustrieren. Stellt euch vor,
ihr braucht in eurem Programm an vielen Stellen einen solchen Primzahltest. Ihr
könntet jedes mal den gesamten benötigten Code hinschreiben, aber das wäre doch
eine Menge Arbeit. Eine Funktion ermöglicht es euch, den Code zum Testen
auszulagern in eine eigene Funktion z.B. \texttt{istprim}. Immer, wenn ihr dann
testen wollt, ob eine Zahl \texttt{n} eine Primzahl ist, könnt ihr dann einfach
schreiben \texttt{istprim(n)} und ruft damit die Funktion auf. Das spart
Schreibarbeit.

Stellt euch vor, irgendwann ist euch dann ein Test mittels Probedivision nicht
mehr schnell genug (weil ihr z.B. immer größere Zahlen testen wollt) und ihr
wollt auf einen schnelleren Test umsteigen. Ohne Funktionen müsstet ihr jedes
Vorkommen eines Primzahltests in eurem Code einzeln durch den neuen Test
ersetzen. Mit Funktionen gibt es genau eine Stelle, an der ihr den Code
austauschen müsst und all die Stellen, an denen ihr ihn benutzt, übernehmen
automatisch die neue Implementation.

Zuletzt haben wir in der letzten Lektion einiges über Lesbarkeit von Quellcode
gelernt. Lesbarkeit erhöht Wartbarkeit und reduziert die Möglichkeiten, Fehler
zu machen. Was stellt ihr euch lesbarer vor, eine Datei, in der einige tausend
Zeilen Code stehen, ohne sichtbare Struktur und einfach nur als Folge von
Anweisungen, oder mehrere kleine Datein, mit beschreibenden, verständlichen
Namen, in denen jeweils mehrere Fun¡tionen, ebenfalls mit verständlichen Namen
stehen, das alles schön aufgeteilt nach Themengebiet? Funktionen erhöhen
Lesbarkeit und helfen, den Code zu strukturieren, was ihn verständlicher macht.

Das wohl wichtigste Beispiel für eine Funktion habt ihr bereits kennengelernt:
Die \texttt{main}-Funktion. Diese spezielle Funktion ist der Eintrittspunkt für
euer Programm. Der Code der \texttt{main}-Funktion ist der erste, der läuft,
sobald die \texttt{main} fertig abgearbeitet ist, beendet sich euer Programm.
Damit haben wir auch schon die Wesentlichen Eckpunkte, wie eine Funktion
syntaktisch auszusehen hat.

Das \texttt{int main()} heißt die \emph{Signatur} der Funktion. Die Signatur
enthält immer den Datentyp, den die Funktion ausgeben soll (in diesem Fall
\texttt{int}, also eine ganze Zahl), einen Namen (in diesem Fall der spezielle
Name \texttt{main}) sowie Parameter, die Eingabewerte (in diesem Fall gibt es
keine Parameter, wir werden aber gleich sehen, wie eine Funktion auszusehen
hat, die Parameter entgegennimmt). An die Signatur schließt sich direkt eine
öffnende geschweifte Klammer, dann kommen alle Anweisungen, aus denen die
Funktion bestehen soll, dann eine schließende geschweifte Klammer.

Um das Ergebnis einer Funktion zurückzugeben, benutzen wir \texttt{return}. Wir
haben das bereits einmal gesehen, wo wir es benutzt haben, um unser Programm zu
beenden -- das funktionierte in dem Fall, weil wir uns in der \texttt{main}
befanden und wie gesagt beendet sich das Programm, sobald die \texttt{main}
fertig ist. Mit Funktionen könnte ein Programm, was eine Zahl darauf testet,
eine Primzahl zu sein, so aussehen:
\inputcpp{funktion.cpp}

Parameter geben wir also in den Klammern an, jeweils so, wie wenn wir eine
Variable erstellen würden, mit einem Typ und einem Namen. Wenn wir mehrere
Parameter brauchen, können wir sie mit Kommata trennen. Dann müssen wir auch,
wenn wir die Funktion benutzen, die einzelnen Eingaben mit Kommata trennen.

\textbf{Praxis:}
\begin{enumerate}
    \item Schreibt eine Funktion, die einen \texttt{std::string} entgegen nimmt
        und einen \texttt{int} zurück gibt. Die Funktion soll -- so wie wir es
        bisher schon in zig Programmen gemacht haben -- den eingegebenen String
        ausgeben und dann einen String von der Nutzerin einlesen. Denkt euch
        einen sprechenden Namen für sie aus.
    \item Passt \texttt{funktion.cpp} so an, dass es eure Funktion benutzt.
    \item Kompiliert das angepasste Programm und lasst es im debugger Schritt
        für Schritt durchlaufen.
\end{enumerate}

\textbf{Spiel:}
\begin{enumerate}
    \item Vertauscht in \texttt{funktion.cpp} die Funktion \texttt{istprim} mit
        der Funktion \texttt{main} (verschiebt also die gesamte Funktion
        \texttt{istprim} an das Ende der Datei). Versucht, die Datei zu
        kompilieren. Was ist die Fehlermeldung des Compilers?
    \item Verschiebt die Funktion \texttt{istprim} \emph{in} die
        \texttt{main}-Funktion (also irgendwo nach der öffnenden geschweiften
        Klammern, aber vor die dazu gehörige schließende). Versucht, die Datei
        zu kompilieren. Was ist die Fehlermeldung des Compilers?
    \item Schaut euch eure bisherigen Lösungen an. Findet ihr noch häufiger
        Stellen, an denen ihr einzelne Teilprogramme in Funktionen auslagern
        könnt?
\end{enumerate}
