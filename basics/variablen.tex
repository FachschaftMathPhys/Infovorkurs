\lesson{Variablen}

Das Programm \texttt{variablen.cpp} erzählt von ihrem/seinen????? Tag.
Compilier es und guck dir die Ausgabe an.

\inputcpp{variablen.cpp}

Da immer wieder das gleiche Wort \glqq{}wunderbar\grqq{} benutzt wird, wurde es in eine sogenannte \emph{Variable} ausgelaggert.
Eine Variable ist ein Name zusammen mit einem Wert.
Im Programm selbst ist es dann so, als würde der Wert an der Stelle des Namens stehen.

Eine Variable kann sich im Laufe des Programmes verändern.
Durch hinzufügen der Zeile \cppinline{beschreibung = "langweilig"} wird hinter dieser Zeile anstelle von \glqq{}wundervoll\grqq{} nun \glqq{}langweilig\grqq{} ausgegeben.
Ähnlich kann wie in \texttt{helloyou.cpp} der Wert von Variablen durch \cppinline{std::cin >> beschreibung} die Userin eingeben werden.

Variablen haben immer einen bestimmten \emph{Datentypen}.
In unserem Beispiel handelt es sich um \cppinline{std::string}.
Der Datentyp wird bei dem Erstellen -- also der ersten Zuweisung -- vor dem Namen angeben.
Dieser wird benötigt, damit der Computer weiß, um was für eine Art Wert es sich handelt -- ein Text sollte anders behandelt werden als eine Zahl.
Beispielsweise kann man zwei Zahlen miteinander multiplizieren, für Texte ergibt das allerdings keinen Sinn.
In der Lektion Arithmetik lernen wir mehr über Zahlen und deren Eigenheiten.

\textbf{Spiel:}
\begin{enumerate}
    \item Was passiert, wenn ihr euch im Namen einer Variable „vertippt“?
    \item Definiert euch zwei \texttt{std::string} Variablen, weißt ihnen
        irgendwelche String-literals zu und versucht, sie zu addieren und das Ergebnis auszugeben.
    \item Was passiert, wenn ihr eine \texttt{std::string} Variable definiert,
        ihr aber nichts zuweist und dann versucht, sie auszugeben?
\end{enumerate}
