\lesson{Variablen}

Wir wollen uns jetzt Zeilen 7 bis 10 von \texttt{helloyou.cpp} anschauen:
\cppsect{helloyou.cpp}{7}{10}

Bisher haben wir uns nicht sehr darum gekümmert, wie genau dieser Teil
funktioniert. Das wollen wir nun korrigieren. Beginnen wir mit Zeile 7.

Was hier passiert ist, dass eine Variable definiert wird. Eine Variable ist im
Wesentlichen ein reservierter Bereich im \emph{Hauptspeicher}, in dem ihr Daten
ablegen könnt und der einen bestimmten Namen (in diesem Fall \texttt{eingabe})
hat. Damit der Computer weiß, wie dieser Speicherbereich zu interpretieren ist
(wir erinnern uns: Der Computer kennt keine Buchstaben oder Zahlen, nur „an“
und „aus“), hat jede Variable einen Typ (in diesen Fall \texttt{std::string},
was einfach eine Folge von Buchstaben ist).

Um einen string anzulegen gibt es noch eine andere Methode: So genannte
\emph{String-literals}. Das ist z.B. das \verb|"Hallo "| in Zeile 10.
Überall, wo ihr ein String-literal benutzen könnt, könnt ihr auch eine
String-Variable benutzen (und umgekehrt)

Um Zeile 10 in Gänze zu verstehen, stellt euch die Pfeile (\verb|<<|) am Besten
als „Schiebeoperatoren“ vor, die das, was auf der Rechten Seite steht in das,
was auf der linken Seite steht „Schieben“. \texttt{std::cout} ist selbst eine
Variable, die bestimmte Dinge tut, wenn etwas „in sie hineingeschoben wird“,
nämlich, es auf der Konsole auszugeben.

Wir schieben also zunächst den String \verb|"Hallo "| auf die Konsole,
anschließend den String, der in der Variable \texttt{eingabe} gespeichert ist,
anschließend ein \texttt{std::endl}, was ein (Betriebssystem abhängiger)
Zeilenumbruch ist.

Was jetzt noch fehlt ist offensichtlich Zeile 8. Das ist, wo wir tatsächlich
eine Usereingabe holen. Wieder stellen wir uns die Pfeile als Schiebeoperatoren
vor, sie schieben jetzt etwas aus \texttt{std::cin} nach \texttt{eingabe}.
Genauso, wie \texttt{std::cout} etwas formatiert auf die Konsole schiebt, zieht
\texttt{std::cin} etwas formatiert aus der Konsole, je nachdem, wo man es
hinschiebt. In diesen Fall schieben wir es in einen String, es wird also ein
String (separiert durch Leerzeichen oder Zeilenumbrüche) gelesen und in die
Variable eingabe geschoben.

Statt uns Strings aus der Konsole zu ziehen, können wir ihnen auch direkt
String-Literals \emph{zuweisen}, wie es hier passiert:
\inputcpp{strings.cpp}

\textbf{Praxis:}
\begin{enumerate}
    \item Was passiert, wenn ihr nach Zeile 8 eine weitere Zuweisung an
        \texttt{hallo} macht?
    \item Definiert euch eine weiter \texttt{std::string} Variable und lest ein
        weiteres Wort darin ein (vielleicht ein Nachname?)
\end{enumerate}

\textbf{Spiel:}
\begin{enumerate}
    \item Was passiert, wenn ihr euch im Namen einer Variable „vertippt“?
    \item Definiert euch zwei \texttt{std::string} Variablen, weißt ihnen
        irgendwelche String-literals zu und versucht, die Summe von beiden
        Strings auszugeben.
    \item Was passiert, wenn ihr eine \texttt{std::string} Variable definiert,
        ihr aber nichts zuweist und dann versucht, sie auszugeben?
\end{enumerate}
