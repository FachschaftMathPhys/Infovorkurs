\lesson{Schleifen}

Wir können mit bedingten Anweisungen den Kontrollfluss schon hilfreich
beeinflussen. Aber nicht alle Dinge, die wir unseren Computer anweisen wollen
zu tun, können wir alleine mit bedingten Anweisungen ausdrücken. Wir können
zwar zum Beispiel testen, ob eine Zahl, eine andere teilt. Was aber, wenn wir
testen wollen, ob eine Zahl eine Primzahl ist? Wir könnten jetzt beginnen, jede
Menge bedingter Anweisungen zu machen, „ist die Zahl durch 2 teilbar, wenn ja,
dann ist es keine, sonst teste, ob sie durch 3 teilbar ist, wenn ja, dann ist
es keine, sonst teste, ob sie durch 5 teilbar ist, wenn ja, dann ist es
keine\dots“, aber es sollte offensichtlich sein, dass wir so nur endlich viele
Teilbarkeiten überprüfen können. Wir müssen zwar für jede Zahl nur endlich
viele Teiler überprüfen, aber wenn die Zahl von der Nutzerin eingegeben wird,
wissen wir im Voraus nicht, wie viele das sind!

Für solche Aufgaben wurden Schleifen erfunden. Sie sind ein Mittel, um eine
Menge von Anweisungen häufig auszuführen, solange eine von uns fest gelegte
Bedingung erfüllt ist. Wenn wir zum Beispiel testen wollen, ob eine Zahl eine
Primzahl ist, wäre ein einfacher Algorithmus die so genannte Probedivision:
Gehe von 2 aufwärts alle Zahlen (die kleiner sind, als die Eingabe) durch,
teste, ob sie die Eingabe teilen -- wenn ja, dann handelt es sich nicht um eine
Primzahl. Haben wir alle Zahlen durchprobiert ohne Erfolg, muss es sich um eine
Primzahl handeln.  Wir können das wieder in einem Kontrollflussdiagramm
ausdrücken ($n$ ist dabei die zu testende Zahl, $i$ ist der Teiler, den wir
gerade testen wollen):

\begin{center}
    \begin{tikzpicture}[auto, node distance=3cm,>=latex']
        \tikzstyle{block} = [draw, fill=blue!20, rectangle, minimum height=3em, minimum width=6em]

        \node [block] (start) {$i = 2$};
        \node [block, right of=start] (cond) {$i < n$?};
        \node [block, right of=cond, node distance=3.5cm] (if) {$i\mid n$?};
        \node [block, right of=if, node distance=4cm] (nope) {$n$ keine Primzahl};
        \node [block, below of=if, node distance=2cm] (incr) {$i \leftarrow i+1$};
        \node [block, above of=nope, node distance=2cm] (yipp) {$n$ Primzahl};

        \draw [->] (start) -- node {} (cond);
        \draw [->] (cond) -- node {ja} (if);
        \draw [->] (cond) |- node [near end] {nein} (yipp);
        \draw [->] (if) -- node {ja} (nope);
        \draw [->] (if) -- node {nein} (incr);
        \draw [->] (incr) -| node {} (cond);
    \end{tikzpicture}
\end{center}
Das Besondere an Schleifen ist, dass sie geschlossene Kreise zum
Kontrollflussdiagramm hinzufügen. Das erlaubt es uns, die gleiche Anweisung
beliebig oft zu wiederholen.

Wenn wir dieses Kontrollflussdiagramm in \Cpp gießen, sieht dies so aus:
\inputcpp{prim.cpp}

Wie wir sehen, sind Schleifen auch nicht viel schwieriger zu handhaben, als
bedingte Anweisungen. Statt \texttt{if} schreiben wir nun \texttt{while}, sonst
ändert sich am Quellcode nicht viel.

Als kleine Nebenbemerkung sei hier gestattet, dass ihr hiermit nun alle Dinge
kennengelernt habt, um \emph{Turing-vollständig} programmieren zu können, d.h.
ihr könnt alleine mit den Mitteln, die ihr bisher kennen gelernt habt.
\emph{jede} mögliche Berechnung anstellen!

\textbf{Praxis:}
\begin{enumerate}
    \item Versucht, die Arbeitsweise eines debuggers zu simulieren. Geht selbst
        den Quellcode Zeile für Zeile durch, überlegt euch, was die Zeile tut
        und welchen Inhalt die Variablen haben. Überlegt euch dann, wohin der
        Computer (bei Kontrollflussstrukturen) als nächstes springen würde.
        Wenn ihr nicht weiter wisst, kompiliert das Programm für den debugger,
        startet es im debugger und geht es durch.
    \item Warum funktioniert das Programm für den Fall $n = 2$?
    \item Schreibt selbst ein Programm, welches eine Zahl von der Nutzerin
        entgegennimmt und dann bis zu dieser Zahl hochzählt.
    \item Modifiziert euer Programm, sodass es von dieser Zahl bis zu 0
        hinunterzählt.
\end{enumerate}

\textbf{Spiel:}
\begin{enumerate}
    \item Das Programm funktioniert noch nicht korrekt, wenn man 1 eingibt
        (denn 1 ist keine Primzahl). Modifiziert es, sodass es auch für 1
        funktioniert.
    \item Kompiliert \texttt{whiletrue.cpp} und führt es aus. Was beobachtet
        ihr? Warum? (Ihr könnt das Programm abbrechen, indem ihr
        \texttt{Strg+C} drückt)
\end{enumerate}

\inputcpp{whiletrue.cpp}
