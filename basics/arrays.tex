\lesson{Arrays}

Als nächstes wichtiges Konzept in \Cpp werden wir uns \emph{Arrays} anschauen.
Arrays sind eine Möglichkeit, mehrere Elemente des gleichen Typs zusammen zu
fassen. Statt also einer Stelle im Speicher, an der ein \texttt{int} liegt,
habt ihr einen ganzen Speicherbereich, in dem 100 (oder eine beliebige andere
Anzahl an) \texttt{int}s liegen.

Die Elemente in einem Array sind durchnummeriert, man nennt die Nummer eines
Arrayelements seinen \emph{Index}. Das erste Element hat den Index 0, das
zweite den Index 1 und das 100te hat den Index 99 -- Vorsicht also, der höchste
Index in einem Array mit 100 Elementen ist 99, nicht 100! Um ein Array zu
definieren, schreibt ihr hinter seinen Namen eine eckige Klammer auf, die
Anzahl an Elementen, die es enthalten soll, und eine eckige Klammer zu. Auf ein
bestimmtes Arrayelement zuzugreifen könnt ihr tun, indem ihr seinen Index in
eckigen Klammern hinter den Namen schreibt. Folgendes Programm macht
hoffentlich die Syntax klar:
\inputcpp{array.cpp}

Es gibt einige Dinge, zu beachnten, wenn ihr mit Arrays arbeitet. Das
wichtigste ist oben schon genannt -- sich davon verwirren zu lassen, dass
Indizes bei 0 anfangen und aus Versehen über das Array hinaus schreiben oder
lesen ist ein so häufiger Fehler, dass er seinen eigenen Namen bekommen hat:
„Off-by-one error“. Wichtig ist, dass der Compiler diesen Zugriff nicht
verhindern wird! Das ist von daher eine sehr fiese Sache, als dass dieser
Fehler auch beim Ausführen nicht immer Probleme machen wird -- aber manchmal
lässt er auch euer Programm spontan abstürzen in einem so genannten
\emph{segmentation fault}.

Eine Limitation von Arrays, die ihr beachten solltet, ist, dass bereits zur
Compilezeit fest stehen muss, wie viele Elemente sie enthalten sollen. Ihr
könnt also z.B. nicht die Nutzerin fragen, wie viele Elemente in das Array
passen soll, denn dies würde erst zur Laufzeit feststehen (wir werden später
noch Wege um diese Limitation kennen lernen).

Ihr könnt auch Arrays von Arrays (so genannte zweidimensionale Arrays)
erstellen, indem ihr zweimal in eckigen Klammern die Größe des Arrays
hinschreibt. Die erste Größe gibt dann die Anzahl der Zeilen an, die zweite die
Anzahl der Spalten. Auch beim Zugriff auf Arrayelemente müsst ihr dann zwei
Indizes angeben. Wir werden dies später noch nutzen, hier sei erst einmal nur
die generelle Möglichkeit genannt.

\textbf{Praxis:}
Wir wollen die Seite \url{http://www.ich-kann-mich-nicht-entscheiden.de/}
nachmachen und eine Entscheidungshilfe programmieren, die aus mehreren von der
Nutzerin gegebenen Möglichkeiten eine per Zufall auswählt.

\begin{enumerate}
    \item Schreibt zunächst ein Programm, welches ein Array aus 10 Strings
        erstellt und die Nutzerin 10 mal nach einer Antwortmöglichkeit fragt
        und die gegebenen Antworten nacheinander in das Array schreibt.
    \item Fügt nun die Möglichkeit zu, weniger Antworten anzugeben. Dazu könnt
        ihr zum Beispiel zuerst fragen, wie viele Antwortmöglichkeiten es geben
        soll und dann so oft fragen (und natürlich einen Fehler ausgeben, wenn
        es mehr als 10 Antworten geben soll).
    \item Ihr könnt dann (so wie in dem Programm oben) eine Zufallszahl
        erzeugen. Um sicher zu gehen, dass sie nicht zu groß wird, könnt ihr
        den Rest bei Teilung durch Anzahl der eingegebenen Antworten nehmen
        (sind z.B. 7 Antworten angegeben und die Zufallszahl ist 25778, so wäre
        der resultierende Index \texttt{25778 \% 7 == 4}). Gebt dann die
        Antwortmöglichkeit aus, die dem zufallsgeneriertem Index
        entspricht.
\end{enumerate}

Sollte euer Programm einmal nicht korrekt kompilieren, denkt daran die
Fehlermeldung sorgfältig zu lesen, damit sie euch Aufschluss über die
Fehlerursache gibt. Sollte euer Programm zwar kompilieren, sich dann aber
komisch verhalten, denkt daran, den debugger zu benutzen und es Schritt für
Schritt durchzugehen, um die Fehlerquelle zu finden. Solltet ihr trotz alledem
nicht weiter kommen, oder nicht wissen, was von euch erwartet wird, fragt einen
von uns.

\textbf{Spiel:}
\begin{enumerate}
    \item Schreibt ein Progamm, welches ein Array beliebiger Größe erstellt und
        dann auf einen Index weit ausserhalb des erlaubten Bereichs schreibt.
        Was beobachtet ihr?\footnote{Es wird natürlich Quark sein was dabei
				rauskommt, es geht hier haupsächlich darum das ihr seht was für
				einen Fehler das gibt}
    \item Implementiert das \emph{Sieb des Eratosthenes}
        \footnote{\url{https://de.wikipedia.org/wiki/Sieb_des_Eratosthenes}},
        wenn ihr noch nicht ausgelastet seid.
        Denkt daran, es initial zu befüllen und denkt euch eine clevere
        Möglichkeit aus, das „Streichen“ zu realisieren.
\end{enumerate}
