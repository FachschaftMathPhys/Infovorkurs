\lesson{Warnings}

Wir haben uns bereits ausführlich mit den Fehlermeldungen des Compilers
auseinander gesetzt. Wir haben auch festgestellt, dass es viele Fehler gibt,
die uns der Compiler durchgehen lässt, die aber im späteren Programmlauf zu
Problemem führen können. Und wir haben den debugger kennengelernt, um die
Ursache solcher Fehler zu finden und sie beheben zu können. Nun wollen wir uns
anschauen, wie wir den Compiler in gewisser Weise „wachsam“ machen können,
sodass er uns auch über Dinge informiert, die zwar keine Fehler sind, aber
möglicherweise zu unerwartetem Verhalten führen können.

\textbf{Praxis:}
\begin{enumerate}
    \item Kompiliert \texttt{warnings.cpp}. Testet das Program mit
        verschiedenen Eingaben. Was beobachtet ihr?
\end{enumerate}
\inputcpp{warnings.cpp}

Fehler wie diese können nicht mehr auftreten, wenn ihr \emph{warnings}
anschaltet. Dies passiert über weitere Optionen, die wir dem Compiler mitgeben.

\textbf{Praxis:}
\begin{enumerate}[resume]
    \item Kompiliert \texttt{warnings.cpp} mittels \texttt{g++ -Wall -o
        warnings warnings.cpp}.
\end{enumerate}

Warnings sehen im Wesentlichen genauso aus, wie Fehler. Der einzige Unterschied
ist, dass statt \texttt{error} in der Zeile ein \texttt{warning} steht und dass
der Compiler zwar die Meldung ausgibt, aber trotzdem ganz normal das Programm
erzeugt. Trotzdem solltet ihr warnings ernst nehmen. Die meisten „ernsthaften“
Programmierer aktivieren warnings, da die meisten davon gefundenen Meldungen
tatsächlich behoben werden sollten.

Ihr könnt mit verschiedenen Parametern beeinflussen, welche warnings euch der
Compiler anzeigt und wie er damit umgeht. Wir wollen hier nur drei nennen:

\begin{description}
    \item[-Wall]
        Aktiviert „alle“ warnings. Tatsächlich stimmt das so nicht, aber wenn
        ihr immer daran denkt, diesen Parameter anzugeben, solltet ihr bereits
        den allergrößten Teil der vom Compiler entdeckbaren Probleme, die ihr
        erzeugt, abfangen können.
    \item[-Wextra]
        Aktiviert noch ein paar warnings (ihr seht, warum „alle“ in
        Anführungszeichen stand). In einigen Fällen sind auch die hier
        aktivierten warnings für euch relevant.
    \item[-Werror]
        Dieser Parameter führt dazu, dass jede warning als Fehler behandelt
        wird, d.h. der Compiler bricht ab, wenn er eine warning produzieren
        würde. Dieser Parameter ist hochumstritten und in der Praxis sollte man
        ihn eigentlich nicht einsetzen. Für Beginner kann er aber hilfreich
        sein, da er von vornherein antrainiert, warnings ernst zu nehmen und
        sie nicht einfach zu ignorieren.
\end{description}

Wenn ihr bei jedem Compilerlauf nun warnings anschalten wollt -- und am Besten auch noch für den debugger, falls ihr ihn braucht -- wird der Befehl zum Kompilieren langsam sehr lang. Für die Dauer des Vorkurses könnt ihr euch mittels

\begin{center}
\texttt{alias\footnote{alias ist ein shell-befehl, der euch eine Reihe von
Anweisungen und Befehlen neu benennen lässt. In diesem Fall ist danach zum
Beispiel der noch nicht existente Befehl \texttt{compile} ein neuer Name für
\texttt{g++ -Vall -Wextra -Werror -O0 -g3}. Beachtet, dass ihr hier genau das
abtippen müsst, was da steht, mit Leerzeichen und allem} compile="g++ -Wall
-Wextra -Werror -O0 -g3"}
\end{center}

ein bisschen Arbeit ersparen. Ein einfaches \texttt{compile -o foo foo.cpp}
wird dann automatisch den Compiler mit allen angegebenen Optionen aufrufen. Den
\texttt{alias} müsst ihr allerdings jedes mal, wenn ihr in der Zwischenzeit ein
Terminal geschlossen habt, neu ausführen, denn er geht bei Schließung eines
Terminals verloren!

\textbf{Praxis:}
\begin{enumerate}[resume]
    \item Mit der warning in \texttt{warnings.cpp} möchte euch der Compiler
        darauf hinweisen, dass ihr hier eine Zuweisung macht, obwohl ein
        Wahrheitswert erwartet wird. Es gibt zwei Möglichkeiten, die warning zu
        beheben: Ihr könnt Klammern um die Zuweisung machen (und dem Compiler
        so sagen, dass ihr euch sicher seid, dass hier eine Zuweisung hinsoll),
        oder ihr könnt aus der Zuweisung einen Vergleich machen. Welche
        Möglichkeit erscheint euch angebracht? Setzt sie um und kompiliert das
        Programm erneut (mit warnings).
    \item In \texttt{warnprim.cpp} haben wir einen Fehler eingebaut. Kompiliert
        das Programm mit warnings und korrigiert ihn.
\end{enumerate}

\inputcpp{warnprim.cpp}

\textbf{Spiel:}
\begin{enumerate}
    \item Die Auswirkungen von warning-Parametern hängen stark von anderen
        Parametern ab. Wir benutzen z.B. häufiger einmal den Parameter
        \texttt{-O0}. Dieser legt das Level an \emph{Optimierung} fest, also
        wie sehr euer Compiler versucht, euren Code umzustrukturieren, sodass
        er sich gleich verhält, aber schneller läuft. \texttt{-O2} würde
        bedeuten, dass relativ stark optimiert wird.

        \texttt{zaehlen.cpp} enthält ein Programm, um bei einem von der Nutzerin eingegebenen Wort zu zählen, wie oft ein bestimmter Buchstabe (zurzeit \texttt{a} vorkommt).
        Kompiliert es mit \texttt{g++ -Wall -O2 -o zaehlen zaehlen.cpp} (probiert es auch einmal ohne \texttt{-O2} oder ohne \texttt{-Wall}) und schaut euch die resultierende Warning an.
        Benutzt gegebenenfalls Google, um heraus zu finden, was sie bedeutet und wie ihr sie beheben könnt.
\end{enumerate}

\inputcpp{zaehlen.cpp}
